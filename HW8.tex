\documentclass{article}
\usepackage[utf8]{inputenc}
\usepackage{geometry}
\usepackage{graphicx}
\usepackage{stackengine}
\usepackage{amsmath}
\usepackage{textcomp}
\usepackage{amssymb}
\usepackage{fixltx2e}
\usepackage{hyperref}
\usepackage{array}
\usepackage{stmaryrd}
\usepackage{tabu}

\geometry{left = 0.5in, right = 0.5in}
\begin{document}

Pablo Acuna

CSCI 4020

Anshelevich

{\centering\ Computer Algorithms Homework 8 \par}

24. Let G = (V, E) be a bipartite graph; suppose its nodes
are partitioned into sets X and Y so that each edge has one end
in X and the other in Y. We define an (a, b)-skeleton of G to be
a set of edges $E'$ $\subseteq$ E so that at most a nodes in X are
incident to an edge in $E'$, and at least b nodes in Y are incident
to an edge in $E'$.

Show that, given a bipartite graph G and numbers a and b,
it is NP-complete to decide whether G has an (a, b)-skeleton. \newline

\textbf{NP:} Given a bipartite graph, set of edges $E'$, and a
(a,b)-skeleton you can go through each edge in $E'$ and place
nodes in two sets such that $X'$ $\subseteq$ X and $Y'$
$\subseteq$ Y.  If $|X'|$ $\leq$ a and $|Y'|$ $\geq$ b then
G has a valid (a,b)-skeleton. \newline

\textbf{Set Cover $\leq_{p}$ (a,b)-skeleton:} Given the parameters of set cover;
$S_1, S_2, ..., S_m$, set U of n elements.  We define a bipartite graph where the nodes
in X correspond to the sets $S_1, S_2, ..., S_m$, and the nodes in Y be the elements in U.
We can create an edge between sets and elements if the set has that element
in it.  We set a = k and b = n.  Put into (a,b)-skeleton blackbox, if yes then
we have a set cover. \newline


\textbf{Proof:} If G has a (a,b)-skeleton $E'$, then the k nodes in X are incident
to the edges in $E'$ correspond to k sets which have all the elements.  Therefore we have a set cover.
If there is a set cover of size k, then taking $E'$ to be the set of all edges incident to the corresponding set nodes gives you
an (a,b)-skeleton.





















\clearpage

Pablo Acuna

CSCI 4020

Anshelevich

{\centering\ Computer Algorithms Homework 8 \par}

26. You and a friend have been trekking through various
far-off parts of the world and have accumulated a big pile
of souvenirs. At the time you weren’t really thinking about
which of these you were planning to keep and which your friend
was going to keep, but now the time has come to divide
everything up.

Here’s a way you could go about doing this. Suppose there
are n objects, labeled 1, 2, ..., n, and object i
has an agreed-upon value $x_i$. (We could think of this,
for example, as a monetary resale value; the case in which
you and your friend don’t agree on the value is something we
won’t pursue here.) One reasonable way to divide things would
be to look for a partition of the objects into two sets, so
that the total value of the objects in each set is the same.

This suggests solving the following Number Partitioning
Problem. You are given positive integers $x_1$, ..., $x_n$;
you want to decide whether the numbers can be partitioned
into two sets $S_1$ and $S_2$ with the same sum:

$$\sum_{x_i \in S_1} x_i  =  \sum_{x_j \in S_2} x_{j} $$

Show that Number Partitioning is NP-complete. \newline

\textbf{NP:} Given two sets $S_1$ and $S_2$ you can find the
sum of all the objects values in each set, if the sums are
equal then there is a number partitioning. \newline

\textbf{Subset Sum $\leq_{p}$ Number Partitioning:}  Given the parameters
for subset sum (W = $\{$$w_1$, $w_2$, ..., $w_n$$\}$, $\bar{W}$) we can make each weight
into a value for an object and we double the list.  Now we make 2$\bar{W}$ as
a value for an object.  Our list of values should look as follows,
X = $\{$$w_1$, $w_1$, $w_2$, $w_2$, ..., $w_n$, $w_n$, 2$\bar{W}$$\}$.  The construction
of this list is 2n +1 which is polynomial.  With this
list we can input it into the Number Partitioning blackbox.  If there is
a number partitioning, then there is a subset sum. \newline

\textbf{Proof:} If there is a subset sum, then there is a $W'$ $\subseteq$ W where...

 $$ \sum_{w_i \in W'} w_i  =  \bar{W} $$

 and using the construction we used for the reduction we can always make two sets
 $S_1$ and $S_2$ such that...

 $$ \sum_{w_i \in W'} w_i + \sum_{w_i \in W'} w_i + \sum_{w_j \notin W'} w_j = 2\bar{W} + \sum_{w_j \notin W'} w_j $$

where the left hand side objects are in $S_1$ and the right is $S_2$.  A similar argument
can be made for the other direction in that if there is a number partitioning, using the summation above we
see that there is a subset sum of size $\bar{W}$





\end{document}
