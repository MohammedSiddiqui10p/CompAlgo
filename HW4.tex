\documentclass{article}
\usepackage[utf8]{inputenc}
\usepackage{geometry}
\usepackage{graphicx}
\usepackage{stackengine}
\usepackage{amsmath}
\usepackage{ textcomp }
\usepackage{ amssymb }
\usepackage{fixltx2e}
\usepackage{hyperref}
\usepackage{array}
\usepackage{tabu}
\geometry{left = 0.5in, right = 0.5in}
\begin{document}
Pablo Acuna

CSCI 4020

Anshelevich

{\centering\ Computer Algorithms Homework 4 \par}

16. There are many sunny days in Ithaca, New York; but this year, as it
happens, the spring ROTC picnic at Cornell has fallen on a rainy day.
The ranking officer decides to postpone the picnic and must notify everyone
by phone. Here is the mechanism she uses to do this.

Each ROTC person on campus except the ranking officer reports to a
unique superior officer. Thus the reporting hierarchy can be described by
a tree T, rooted at the ranking officer, in which each other node v has a
parent node u equal to his or her superior officer. Conversely, we will
call v a direct subordinate of u. See Figure 6.30, in which A is the
ranking officer, B and D are the direct subordinates of A, and C is the
direct subordinate of B.

To notify everyone of the postponement, the ranking officer first calls
each of her direct subordinates, one at a time. As soon as each subordinate
gets the phone call, he or she must notify each of his or her direct
subordinates, one at a time. The process continues this way until everyone
has been notified. Note that each person in this process can only call
direct subordinates on the phone; for example, in Figure 6.30, A would not
be allowed to call C.

We can picture this process as being divided into rounds. In one round,
each person who has already learned of the postponement can call one of his
or her direct subordinates on the phone. The number of rounds it takes for
everyone to be notified depends on the sequence in which each person calls
their direct subordinates. For example, in Figure 6.30, it will take only
two rounds if A starts by calling B, but it will take three rounds if A
starts by calling D.

Give an efficient algorithm that determines the minimum number of
rounds needed for everyone to be notified, and outputs a sequence of
phone calls that achieves this minimum number of rounds. \newline

\textbf{Subproblems}: Our subproblems will all subtrees $T^{'}$ in T. With
this we will define OPT($T^{'}$) to be the minimum number of rounds
for everyone in $T^{'}$ to be notified once the root has the message.
Define $T_{1}$, ..., $T_{k}$ to be all the child subtrees of $T^{'}$. We will also
say that we sorted the child subtrees such that OPT[$T_{1}$] $\geq$ , ...,
$\geq$ OPT[$T_{k}$] \newline

\textbf{Recursion Relation}: OPT($T^{'}$) = $min_{1 \leq j \leq k}$ [j + OPT(j)] \newline

This is a correct relation because are looking at the subtrees direct childern
looking through all of them finding which one has the smallest sum of its
optimal solution and sorted j value which is used to determine which the order in which
each should get called first because biggest subtree. \newline

\textbf{Algorithm}: The base case for this problem will be the leaves, all leave
node subtrees solution will be: OPT[$T^{'}$] = 0.
Building from the base case, we will traverse our way up the tree to the root.
The solution will be the entire tree OPT[T].  To obtain the sequence we can
recursively trace back through the sorted order and print the list of rounds. \newline

\textbf{Analysis}: Since there are n nodes in the tree, there are n subtrees,
and therefore n subproblems.  To compute each subproblem we must sort first which
takes O(nlogn).  And to obtain the optimal solution we have to iterate at most
n-1 previous solutions, therefore the sorting is larger and the total runtime is
O($n^2$logn).


\clearpage

Pablo Acuna

CSCI 4020

Anshelevich

{\centering\ Computer Algorithms Homework 4 \newline\par}

24. Gerrymandering is the practice of carving up electoral districts in
very careful ways so as to lead to outcomes that favor a particular
political party. Recent court challenges to the practice have argued that
through this calculated redistricting, large numbers of voters are being
effectively (and intentionally) disenfranchised.

Computers, it turns out, have been implicated as the source of some of
the “villainy” in the news coverage on this topic: Thanks to powerful
software, gerrymandering has changed from an activity carried out by a
bunch of people with maps, pencil, and paper into the industrial-strength
process that it is today. Why is gerrymandering a computational problem?
There are database issues involved in tracking voter demographics down
to the level of individual streets and houses; and there are algorithmic
issues involved in grouping voters into districts. Let’s think a bit
about what these latter issues look like.

Suppose we have a set of n precincts $P_{1}$, $P_{2}$, . . . , $P_{n}$,
each containing m registered voters. We’re supposed to divide these
precincts into two districts, each consisting of n/2 of the precincts.
Now, for each precinct, we have information on how many voters are
registered to each of two political parties. (Suppose, for simplicity,
that every voter is registered to one of these two.) We’ll say that the
set of precincts is susceptible to gerrymandering if it is possible to
perform the division into two districts in such a way that the same party
holds a majority in both districts.

Give an algorithm to determine whether a given set of precincts is
susceptible to gerrymandering; the running time of your algorithm should
be polynomial in n and m. \newline

\textbf{Subproblems}: OPT[j, p, x, y] = true if it is possible to have
at least x Party A voters in distict 1 and at least y Party A voters in
district 2, while using p of the 1 through j precincts for district 1.
If it is not possible then it is false.  Lets also define the array of
registered Party A voters for each precinct to be $A_{1}$, ...,
$A_{n}$.  This will allow us to make the recursion relation. \newline

\textbf{Recursion Relation}: OPT[j, p, x, y] = OPT[j-1, p-1, diff(x-$A_{j}$), y]
 or OPT[j-1, p, x, diff(y-$A_{j}$)]

 \indent \indent diff(n):

  \indent \indent \indent  if n $\geq$ 0

  \indent \indent \indent \indent  return n

  \indent \indent  return 0

 This is a correct relation because we have two choices for adding a precinct,
 we either add it to distict one or district two.  If we add it to distict 1
 we check if it was possible for the p-1 precincts to have x-$A_{j}$
 voters and the rest have y voters.  Or we will look at not including it in
 distict 1 which is the opposite as the first option. \newline

\textbf{Algorithm}: The base cases will be OPT[j, p, 0, 0] for all j and p,
in this case it is possible to have 0 Party A members in both districts
therefore, OPT[j, p, 0, 0]=true.  All other cells will be false.
We can build this up in increasing j,p,x,y.  They all start at 1 and their
loop ends at n, j, mn/4, and mn/4 respectively. The solution will be at
OPT[n, n/2, nm/4, nm/4].  That is because if it is true then we have at least
half of each district is party A voters. Therefore we have gerrymandering.
We can also run this with Party B afterwards. \newline


\textbf{Analysis}:  We have $n^4$$m^2$ subproblems and from the relation
takes constant time to compute.  Therefore our runtime is O($n^4$$m^2$)
\end{document}
