\documentclass{article}
\usepackage[utf8]{inputenc}
\usepackage{geometry}
\usepackage{graphicx}
\usepackage{ textcomp }
\usepackage{fixltx2e}
\usepackage{hyperref}
\usepackage{array}
\usepackage{tabu}
\geometry{left = 0.5in, right = 0.5in}
\begin{document}
Pablo Acuna


CSCI 4020


Anshelevich


{\centering\ Computer Algorithms Homework 2 \par}

1. Prove the conjectures:

\indent \indent a. The minimum spanning tree of G, with respect to the
edge weights $a_{e}$, is a minimum-altitude connected subgraph

\indent \indent b. A subgraph (V , $E^{'}$) is a minimum-altitude
connected subgraph if and only if it contains the edges of the
minimum spanning tree. \newline



Proof by Contradiction:
Suppose we have a cut $(S, V-S)$ of V, there is an edge $e=(u,v)$ which is
the edge with the minimum altitude value across cut $(S, V-S)$, and
one minimum-altitude connected subgraph $(V,E')$ does not contain
$e$. Suppose the winter-optimal path between $u$ and $v$ in
$E'$ is $P$.  Then $P$ must contain at least one edge $e'$ that is
across the cut. Since $e$ is the minimum-altitude edge across the cut
and edge $e$ and egde $e^{'}$ make a cycle, we have $a_{e'}>a_e$,
and the height of $P$ is larger than or equal to $a_{e'}$,
therefore larger than $a_e$, but the height of winter-optimal path
in the full graph $(V,E)$ is less than or equal to $a_e$ because
edge $e$ makes one path between $u$ and $v$.  This means the height of
winter-optimal path in $(V,E')$ is greater than it is in the full
graph, which contradicts the definition of minimum-altitude
connected subgraph.

Since the minimum spanning tree is composed of edges
with minimum altitude values across cuts, directly yields the result.
Every minimum-altitude connected subgraph contains all the edges
in the minimum spanning tree and because adding edges to a
minimum-altitude connected subgraph always yields another
minimum-altitude connected subgraph, so we also have the result.
Every supergraph of the minimum spanning tree is a minimum-altitude
connected subgraph. Both conjectures in the problem are true.

\clearpage
Pablo Acuna


CSCI 4020


Anshelevich


{\centering\ Computer Algorithms Homework 2 \par}

2.

\indent \indent a. Show that the following algorithm does not correctly
solve this problem, by giving an instance on which it does not return the correct
answer. In your example, say what the correct answer is and also what the algorithm
above finds. \newline

{\centering\ \begin{tabu} to 0.8\textwidth { |X[c]|X[c]|X[c]|X[c]|X[c]|}
 \hline
 Month & 1 & 2 & 3 & 4\\
 \hline
 NY  & 1 & 1 & 2 & 3  \\
\hline
 SF  & 2 & 2 & 1 & 1  \\
\hline
\end{tabu} \newline \par}


With an M = 100, the algorithm in the book will return to you the minimum cost
sequence of [NY, NY, SF, SF].  The cost of this sequence would be 104.  This
is not an optimal solution.  The optimal solution is [SF, SF, SF, SF] with
cost of 6. \newline

\indent \indent b. Give an example of an instance in which every optimal plan must
move (i.e., change locations) at least three times. Provide a brief explanation,
saying why your example has this property. \newline

{\centering\ \begin{tabu} to 0.8\textwidth { |X[c]|X[c]|X[c]|X[c]|X[c]|}
 \hline
 Month & 1 & 2 & 3 & 4\\
 \hline
 NY  & 20 & 70 & 25 & 120  \\
\hline
 SF  & 100 & 10 & 90 & 30  \\
\hline
\end{tabu} \newline \par}


With M = 10, the optimal solution sequence is [NY, SF, NY, SF] with a cost of
105.  The property that the example has is a low cost of moving plus cost at
the moving location be far less than potato staying in the same city. \newline


\indent \indent c. Give an efficient algorithm that takes values for n, M,
and sequences of operating costs $N_{1}$, ..., $N_{n}$ and $S_{1}$,...,$S_{n}$,
and returns the cost of an optimal plan. \newline


Subproblems: L[i][1] = Optimal solution when starting at NY at the ith month


\indent\indent\indent\indent\indent L[i][2] = Optimal solution when starting
at SF at the ith month\newline


Recursion Relation: L[i][1] = min(L[i-1][2] + M, L[i-1][1] + $N_{i}$)


\indent\indent\indent\indent\indent \indent \indent L[i][2] =
min(L[i-1][1] + M, L[i-1][2] + $S_{i}$) \newline

This relation works because the optimal solution will be the minimum between
staying put at a location or moving and adding the cost of moving, M, to
the optimal solution for the last month at the other city. \newline

Psuedo Code:

L[n][2]

L[1][1] = $N_{1}$

L[1][2] = $S_{1}$

for i = 2 to n


  \indent \indent  L[i][1] = min(L[i-1][2] + M, L[i-1][1] + $N_{i}$)


  \indent \indent  L[i][2] = min(L[i-1][1] + M, L[i-1][2] + $S_{i}$)

return min(L[n][1], L[n][2]) \newline


\end{document}
